\chapter{Introduction}
\label{sec:intro}

\textbf{Thesis Statement}: Malware on Mobile Operating Systems, especially Android, is better understood when not just analyzing the capabilities of an application, but the expectations the user has as to how it utilizes those capabilities as well.\\


The rise of smartphones in the last 6 years has been nothing short of meteoric. Since the launch of the Apple iPhone in 2007, over \temp{1 billion or so} smartphones have been sold, from over \temp{100 or so}\citetemp{Author, Year} manufacturers. These new devices marked an unprecedented shift in our relationship with computers, becoming the center point for many personal endeavors, and superseding almost all previous computing devices from cell phones, to cameras, to GPS devices, and to most uses of a desktop pc \citetemp{Author, Year}. Indeed, smartphones continue to become the focal point of almost all personal computing, and consequently the operating systems they run become more important and powerful.

Mobile operating systems (mobile OSs), like the PC operating systems of the 1990s, have a few major players that wield the most influence. The two largest operating systems in the mobile area are Android and iOS. Apple’s iOS, made exclusively for the Apple iPhone and iPad, currently runs on over 15\%\citetemp{Author, Year} of all smartphones globally. Google’s Android, released as an open source OS, has many different hardware manufacturers, Samsung, LG, HTC, Motorola, and many more. It currently runs the majority of smartphones globally, with a 75\% marketshare. Some of the less popular, but still significant mobile operating systems are Windows Phone, with 2\%, and Blackberry, with 4\%. 

\temp{Introduce iOS and Android!} Android was started in 2003 by Andy Rubin and Android, Inc, previously the makers of the T-Mobile Sidekick. In 2005 they were acquired by Google Inc, and work continued in secret until it’s first release in 2008 with the HTC Dream.

The diversity of hardware that smartphones were designed to replace, along other constraints and features, requires a mobile OS that’s designed from the ground up to deal with many different challenges than the typical PC OS. Some of the main design challenges for a mobile OS are: 
\begin{smitemize}

\item Small memory footprint, battery conscious, and other resource constrictions

\item Access to a wide variety of personally identifiable information (PII)

\item Access a wide array of hardware

\end{smitemize}
In order to effectively enforce rules on battery consumption, low-latency UI, and personally identifiable information, a new security model was created, centered around the concept of the ‘App’. 

This security model has dramatically changed the nature of mobile software, and in turn, mobile malware. By forcing malware to fit inside of this security sandbox, malware authors must choose to either break out of the box, or work inside of it. This constriction has blurred the definition of mobile malware, and ultimately has profound implications for the user of the mobile device. We will examine these implications, and propose new tools and methods to help understand the nuances of modern malware, as well as provide a framework of tools to detect them.


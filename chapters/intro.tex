\chapter{Introduction}
\label{sec:intro}

%\textbf{Thesis Statement}: Malware, and especially Info Theft Malware, on Mobile Operating Systems, especially Android, is better understood when not just analyzing the capabilities of an application, but the expectations the user has as to how it utilizes those capabilities as well.\\
%\textbf{Thesis Statement}: Android Malware detection is complemented by not just analyzing the capabilities of an app, but the context and use as well.\\
%\textbf{Thesis Statement}: In order to detect Android Malware, the user must understand what is going on behind the scenes of the app. The capabilities of an app are important, but the context and use of these capabilities are as well.\\
\textbf{Thesis Statement}: Using a novel feedback loop, we provide users with a method for understanding the context and use of actions that a mobile app performs, thus allowing users to identify suspicious behavior that violates their trust.\\


The rise of smartphones in the last decade has been unprecedented. Since the launch of the Apple iPhone in 2007, there are now almost 1 billion smartphone users in the world\citep{kpcbinternetreport2012}. These new devices mark an unprecedented shift in our relationship with computers, becoming the center point for many personal endeavors, and superseding almost all previous computing devices including cell phones, cameras, GPS devices, and to most uses of a desktop PC\citep{hua2012introduction}. Smartphones continue to become the focal point of almost all personal computing --- and consequently the operating systems they run become more important and powerful.

\section{Contributions}
In this thesis, we highlight four key areas:
\begin{smitemize}
\item \textit{User-App Agreement}: First, we discuss the challenges of addressing modern mobile malware and the shortcomings of the Android security model. We introduce the User-App Agreement (UAA) --- a way of conceptualizing the trust a user has in actions an application may perform --- as a key component in identifying malicious behavior.
\item \textit{Android Census}: Second, we use a novel dataset, Android Census, to examine the state of Android permissions. We find Android permissions correlate with expected use, but key examples are shown of less than legitimate use. Using a comprehensive set of malware, we cross-examine how the permissions of malware compares with the Android Census. We conclude that malware that targets the user's personal information is the most difficult to detect using this static analysis.
\item \textit{IncognitoWare}: Third, we address the shortcomings of the current malware datasets available to academia and introduce a new dataset of IncognitoWare. This dataset is more representative of current trends in malware and proves to be a great challenge to detect.
\item \textit{AndroMEDA}: Finally, we introduce Android Malware Evaluation Detection and Analysis (AndroMEDA), a set of Android extensions and a companion app built off of the premise of the User-App Agreement. By giving the user more information on the context and use of sensitive system actions, they can evaluate whether they trust those actions and, ultimately, whether the app is acting maliciously or not.
\end{smitemize}





% After untrusted behavior is spotted, the actions can be reported, and knowledge can be spread to all users. All of this makes users more aware of app behavior, and helps mitigate Info Theft Malware on Android.

%Building off the concept of the User-App Agreement, we introduce AndroMEDA. Key parts of the User-App Agreement were previously unnoticeable to the user until AndroMEDA. By giving the user more information on the context and use of permissions, they can evaluate whether they trust those actions, and ultimately whether the app is acting maliciously or not. After untrusted behavior is spotted, the actions can be reported, and knowledge can be spread to all users. All of this makes users more aware of app behavior, and helps mitigate Info Theft Malware on Android.


%This security model has dramatically changed the nature of mobile software, and in turn, mobile malware. By forcing malware to fit inside of this security sandbox, malware authors must choose to either break out of the box, or work inside of it. This constriction has blurred the definition of mobile malware, and ultimately has profound implications for the user of the mobile device. We will examine these implications, and propose new tools and methods to help understand the nuances of modern malware, as well as provide a framework of tools to detect them.


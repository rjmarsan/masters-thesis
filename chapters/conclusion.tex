\chapter{Conclusion and Future Work}
\label{sec:conclusion}
There are several directions in which this work can be extended to provide
better support for the integration of NVBM. 

\textbf{Reliability}\\
In current systems, file systems play an important role in ensuring that data is
not corrupted by device failures. In CDDS-based systems, the OS (along with
userspace libraries) needs to handle failures in NVBM devices.  Though NVBM
technologies are projected to have greater endurance than Flash-based
SSDs~\cite{Zhou09}, the OS will also need to provide support for wear-leveling
algorithms in the persistent virtual memory subsystem. We are investigating how
the integration of wear-leveling with CDDS versioning can lead to benefits in
performance and reliability. 

\textbf{Multicore Architectures}\\
The advent of NVBM as a replacement for disks is one among the many
architectural trends, which are influencing the design and implementation of
next-generation operating systems. Trends in architecture also suggest that
many-core machines with tens to hundreds of cores will be available
soon~\cite{Agarwal07}.  While operating systems like Barrelfish~\cite{Baumann09}
propose OS designs where message passing is used to communicate between
multiple cores, CDDS based systems propose that NVBM is made directly available
to applications.  It remains to be seen whether NVBM allocation and garbage
collection schemes would affect operating systems~\cite{Boyd-Wickizer08}
designed for such many-core machines.

\textbf{Safety}\\
CDDS-based systems currently depend on virtual memory mechanisms to
provide fault-isolation and like other services, it depends on the OS
for safety.  Therefore, while unlikely, placing NVBM
on the memory bus can expose it to accidental writes from rogue DMAs.
In contrast, the narrow traditional block device interface makes it
harder to accidentally corrupt data.  We believe that hardware memory
protection, similar to IOMMUs, will be required to address this
problem.  Given that we map data into an application's address space,
stray writes from a buggy application could also destroy data.  While
this is no different from current applications that \texttt{mmap}
their data, we are developing lightweight persistent heaps that use
virtual memory protection with a RVM-style
API~\citep{Satyanarayanan94} to provide improved data safety.

The end-to-end argument~\cite{Saltzer84} in system design states that
functionality provided at lower layers is redundant when compared to the cost
of providing them. In the context of the persistent storage on NVBM, this work
has shown that low-level abstractions like the block device layer and file
interface are too expensive and that we need to rethink the design of storage
systems for NVBM\@. This work has presented the design and implementation of
Consistent and Durable Data Structures (CDDSs), an architecture that, without
processor modifications, allows for the creation of log-less storage systems on
NVBM\@.  Results from our experiments show that redesigning systems to support
single-level data stores will be critical in meeting the high-throughput
requirements of emerging applications. 

% LocalWords:  CDDSs CDDS versioning multi NVBM

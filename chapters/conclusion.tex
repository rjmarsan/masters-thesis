\chapter{Conclusion and Future Work}
\label{sec:conclusion}
\section{Conclusion}
AndroMEDA helps users understand the context in which their Personally Identifiable Information is used, which allows them to make more informed decisions on whether an app is action maliciously or not. In this paper, we introduced 4 key items:

\subsection{User-App Agreement}
%We analyze the state of Android and Permissions, and conclude the need to addrjess the User-App Agreement, which is a framework for consenting and trusting specific actions an app may take. It lies at the heart of classifying Info Theft Malware. Android Permissions address the capabilities of an app, but fail to address the context and use of those capabilities, which are cruicial to trusting an app's actions.

We analyzed the current Android security framework: The Permission System, and found it's main flaws were it's lack of addressing context and use, which we generalize into the User-App Agreement - a framework for consenting and trusting specific actions an app may take. Whereas Android Permissions exceeded at defining general capabilities of an app, and these capabilities go a long way in shaping the User-App Agreement, they fail at addressing the context in which the permissions are used, and what they are used for.

\subsection{Android Census}
To perform a full analysis of the current state of Android Permissions, we use a novel dataset, Android Census. By analyzing more than 80\% of apps in the Google Play Store, we are able to better understand the interrelationship of permissions and expected behavior. We produce key insights as to the popularity of apps vs their PII permissions, and when apps deviate from their expected behavior - potentially violating the User-App Agreement. We then analyze a comprehensive malware dataset using the same techniques, and find that many types of malware can be identified purely by it's permission fingerprint. We conclusively show the connection between Permissions and Expected Behavior are present, but not strong enough to differentiate between Info Theft Malware and many popular apps.

\subsection{AndroMEDA}
Building off the concept of the User-App Agreement, we introduce AndroMEDA. Key parts of the User-App Agreement were previously unnoticable to the user until AndroMEDA. By giving the user more information on the context and use of permissions, they can evaluate whether they trust those actions, and ultimately whether the app is acting maliciously or not. After untrusted behavior is spotted, the actions can be reported, and knowledge can be spread to all users. All of this makes users more aware of app behavior, and helps mitigate Info Theft Malware on Android.

\subsection{IncognitoWare}
We highlighted the need for more modern Android malware datasets. To address this, we introduced a dataset of IncognitoWare - repackaged trusted apps with additional malicious behavior embedded. We believe this is a more comprehensive look at Android security because it highlights the need to understand context in order to identify malicious behavior.

\section{Future Work}
\label{sec:futurework}
AndroMEDA is, ultimately, not a silver bullet at detecting all Android malware. Projects like TaintDroid and TISSA provide functionality that would greatly enhance the data gathering abilities and response options of AndroMEDA. Integrating TISSA would allow users or AndroMEDA to temporally block access to sensitive data, while not blocking access to the same data at a later time, when the user trusts the action. Overall, adding more API instrumenting would prove useful, as projects like TapLogger provide future challenges.

A main focus of AndroMEDA is providing a feedback loop to the user, and to that end, visualization is an important area of future work. Visualizing the wealth of information in a concise way that avoids user fatigue is a main challenge, and subject to future study.

AndroMEDA could also benefit greatly from the probablistic modeling of pBMDs and Crowdroid, in correlating user action with permission behavior. These would not replace the need to alert the user, but rather be able to better dictate when to put different classes of alerts to the user, as ultimately the decision of what is malware is up to them.

The weath of data in Android Census was also not fully explored. We are currently interested in seeing if specific keywords in user reviews correlate with malicious software, or other problematic apps. Many more areas of metadata, like the description, developer, etc, can be further explored, to see if it gives additional insight into the nature of malware on Android.

Finally, the concept of the User-App Agreement introduced in this paper can be expanded upon greatly. User Studies of trust in specific actions can greatly increase our understanding of what actions users tend to trust, and when they are untrustworthy. These same user studies could also help show the effectiveness of AndroMEDA in spotting Info Theft Malware.
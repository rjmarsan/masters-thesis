\chapter{Related Works}
\label{sec:relatedworks}

%\temp{THIS WILL/MAY BE GREATLY EXPANDED UPON LATER}

\section{Android Extensions}
AndroMEDA is far from the first to attempt to address the issue of malware on Android. As early as 2009, frameworks like SAINT\citep{ongtang2012semantically} built off of the Android Permission Framework by implementing runtime policies the user could define per-app. Later on, projects like TISSA\citep{zhou2011taming} built off of this by implementing varying levels of obfuscated data. When an app running on TISSA requests access to the contacts database, TISSA can either provide the app the full database, some limited portion, some anonymized portion, or outright returning no information. Not all PII was covered in this, nor were any temporal rules established: the system behavior for a specific database of PII for an app was consistent across requests. These events were not shown to the user either, something addressed by TaintDroid\citep{enck2010taintdroid}.

TaintDroid is a novel extension for Android, focused on flow detection of PII. By modifying the low level VM of Android, Dalvik, variables are tainted once they access PII. This taint flows throughout the system, and when the variable reaches a designated exit location, the event is logged and alerted to the user. Finally, YAASE\citep{russello2011yaase} is a relatively new security extension, but combines many aspects of TaintDroid and TISSA, to become an extremely powerful way of detecting information flow and prevention.

There are several main shortcomings in all of these frameworks. First off, they all require significant modification of the Android codebase, thus creating very difficult work for developers who seek to incorporate these extensions into their Android OS. Performance impact, although somewhat negligible by many accounts, certainly plays a role in a decision to incorporate such security extensions.

Another important shortcoming is the complex rulesets required. Most build off of the Permission Framework, adding additional rules the user may configure. However, a study by Berkeley suggests only 17\% paid attention to permissions at install time, and only 3\% correctly remembered them later\citep{felt2012android}. Clearly, tasking the user with more work is not the right approach. Looking at malware through the context of the UAA, all but TaintDroid fall short in one key regard: alerting the user of suspicious behavior.

Several frameworks have touched upon the concept of the UAA. Andromaly\citep{shabtai2012andromaly}, developed in 2010, pBMDS\citep{xie2010pbmds} and Crowdroid\citep{burguera2011crowdroid} all attempt to classify malware based upon its interaction with the user. However, none actually ask input from the user - a fundamental flaw that limits the ability to adapt to the user's specific UAAs.

\section{Android Sandboxes}
On the other side of malware detection is automated malware detection. The major project in this regard is Google's Bouncer\citep{googlebouncer}. Introduced in 2011, Google Bouncer is a system that runs malware in a highly-observed sandbox, and watches for suspicious behavior. Since its release, it has been the subject to quite a bit of criticism\citep{mansfield2012android}, with researchers finding over 20 ways to circumvent it.

Along similar lines, TrendMicro provides its solution, App Reputation\citep{trendmicroappreputation}. It runs apps in a ``cloud'' sandbox, watching for connections to suspicious websites, as well as other monitoring. A research project, Paranoid Android\citep{portokalidis2010paranoid}, runs in a similar vein, monitoring apps in a sandboxed Android OS. However, in order to get accurate information on app behavior, actions must be recorded.

However, most all of the frameworks listed above, especially the sandbox tools, assume a clear ruleset to be established classifying malicious behavior vs benign behavior. They require a clear line to be established as to what constitutes trusted behavior vs untrusted. Unfortunately, this is counter to the concept of the UAA, where every individual action has a complex set of rules that result in an acceptable behavior vs unacceptable. Users may also have vastly different rules for what constitutes malicious behavior. Apps that send the Unique Device ID (UDID) and location information to ad networks might be malicious to some users, and perfectly normal to others.


\section{Conclusion}
%This should be the actual conclusion
The main counter argument to all of these frameworks is the example of SoundComber: it records in the background, looking for PII. Some frameworks and policies may immediately flag this as malware, but one can conceive of perfectly benign apps that would follow this exact same formula: dictation apps that transcribe speech to text over long portions of time, or audio broadcasting utilities. Apps that take all PII and upload it to a server may be classic malware, or simply a backup utility. Context and Use, and more generally the UAA, is an extremely important part of malware detection that's missing from modern Android security frameworks.

\chapter{Related Works}
\label{sec:relatedworks}

%\temp{THIS WILL/MAY BE GREATLY EXPANDED UPON LATER}

\section{Android Extensions}
AndroMEDA is far from the first to attempt to address the issue of malware on Android. As early as 2009, frameworks such as SAINT\citep{ongtang2012semantically} built off of the Android Permission Framework by implementing runtime policies the user could define per app. Later on, projects like TISSA\citep{zhou2011taming} built off of SAINT by implementing varying levels of obfuscated data. When an app running on TISSA requests access to the contacts database, TISSA can either provide the app with the full database, a limited portion, an anonymized portion, or no information. Not all PII was covered in their framework, nor were any temporal rules established: the system behavior for a specific database of PII for an app was consistent across requests. These events were not shown to the user either, unlike TaintDroid\citep{enck2010taintdroid}.

TaintDroid is a novel extension for Android, focused on flow detection of PII. By modifying the low-level VM of Android --- Dalvik --- variables are tainted once they access PII. This taint flows throughout the system, and when the variable reaches a designated exit location, the event is logged and alerted to the user. Finally, YAASE\citep{russello2011yaase} is a relatively new security extension, but combines many aspects of TaintDroid and TISSA to become an extremely powerful way of detecting information flow and prevention.

There are several main shortcomings in all of these frameworks. First off, they all require significant modification of the Android codebase, thus creating very difficult work for developers who seek to incorporate these extensions into their Android OS. Performance impact, although by many accounts is somewhat negligible, certainly plays a role in a decision to incorporate such security extensions.

Second, the complex rulesets required build off of the Permission Framework, adding additional rules the user may configure. However, a study by Berkeley suggests only 17\% of users paid attention to permissions at install time, and only 3\% correctly remembered them later\citep{felt2012android}. Clearly, tasking the user with more work is the wrong approach. Looking at malware through the context of the UAA, all but TaintDroid fall short in one key regard: alerting the user of suspicious behavior.

Several frameworks have touched upon the concept of the UAA. Andromaly\citep{shabtai2012andromaly} (developed in 2010), pBMDS\citep{xie2010pbmds}, and Crowdroid\citep{burguera2011crowdroid} all attempt to classify malware based upon its interaction with the user. However, none actually ask immediate input from the user --- a fundamental flaw that limits the ability to adapt to the user's specific UAAs.

\section{Android Sandboxes}
On the other side of malware detection is \textit{automated} malware detection. The major project in this regard is Google's Bouncer\citep{googlebouncer}. Introduced in 2011, Google Bouncer is a system that runs malware in a highly-observed sandbox, and watches for suspicious behavior. Since its release, it has been highly criticized\citep{mansfield2012android}, with researchers finding over twenty ways to circumvent it.

Similarly, TrendMicro provides its solution, App Reputation\citep{trendmicroappreputation}. It runs apps in a ``cloud'' sandbox, watching for connections to suspicious websites, as well as other monitoring. A research project, Paranoid Android\citep{portokalidis2010paranoid}, runs in a similar vein, monitoring apps in a sandboxed Android OS. However, in order to get accurate information on app behavior, actions must be recorded.

Most all of the frameworks listed above, however, especially the sandbox tools, assume a clear ruleset to be established classifying malicious behavior versus benign behavior. They require a clear line to establish what constitutes trusted behavior versus untrusted. Unfortunately, this counters the concept of the UAA, in which every individual action has a complex set of rules that result in an acceptable behavior versus unacceptable. Users may also have vastly different rules for what constitutes malicious behavior. Apps that send the Unique Device ID (UDID) and location information to ad networks might seem malicious to some users, and perfectly normal to others.


\section{Conclusion}
%This should be the actual conclusion
The main counter argument to all of these frameworks is the example of SoundComber: it records audio in the background, looking for PII. Some frameworks and policies may immediately flag this as malware; but one can conceive of perfectly benign apps that follow this exact same formula: dictation apps that transcribe speech to text over long portions of time, audio broadcasting utilities, etcetera. Apps that take all PII and upload it to a server may be classic malware --- or simply a backup utility. Context and use --- and more generally the UAA --- are an extremely important part of malware detection that's missing from modern Android security frameworks.

\chapter{Market Analysis}
\label{sec:market}

In order to evaluate AndroMEDA effectiveness in identifying malware, it's important to evaluate the ability to identify malware without it. As previously discussed in \ref{sec:permissions}, Permissions are the single security measure that defends an Android user from malicious software. We hope to show a fundamental disconnect between Permissions and UAA, as conclusive evidence that Permissions should not be the sole security system on Android. In order to study this, we must examine the how Permissions reflect app behavior, and if known malware can be identified with Permissions alone.



\begin{table*}[t]
\begin{small}
\begin{tabular}{r|l}
Metadata & Description \\
\hline

\textit{App Name} & The name of the app, e.g. Google Maps  \\
\textit{App Developer} & The name of the developer, e.g. Google  \\
\textit{Android Version} & The lowest compatible Android version  \\
\textit{Number of Installs} &  The total number of installs. Given as a range. \\
\textit{Description} &  A long (3000 word max) description of the app. \\
\textit{Reviews} &  The user reviews of the app. \\
\textit{Overall Rating} &  The overall rating of the app, from 1 to 5. A user does not need to write a review to leave a rating. \\
\textit{Requested Permissions} &  The list of all the permissions the app requests. \\

\end{tabular}
\end{small}
%\vspace{-0.2in}
\caption{Metadata from \temp{AndroidMarketDB}}
\label{tab:marketmetadata}
%\vspace{-0.1in}
\end{table*}


\section{Market Dataset}
To do a broad scale study of Android permissions, we introduce a novel dataset: \temp{AndroidMarketDB}. \temp{AndroidMarketDB} contains metadata from \temp{80\%} of all known Android Apps in the GPStore. This metadata, described in \ref{tab:marketmetadata}, is rich in contextual information about an app. For a period of 1 month in May-June 2012, the GPStore was scanned twice a day. This dataset provides detailed metadata for all areas of the Android market, not just the top percent. Despite the large amount of metadata available, we will focus on the Permissions, leaving most other fields for Future Work\ref{sec:futurework}.



\section{Global Permission Analysis}
We first look at the global trends of Permissions in the GPStore. Ideally, we find that access to sensitive PII, and dangerous operations, only is requested by a handful of apps that truely need them. This both increases the user's ability to understand how an app behaves, increases the user's trust in the Permission system, and demonstrates a strong connection between Permission Fingerprint and User-App Agreement. Ultimately, however, we expect to find a system that falls short of these goals.

\begin{figure}[t]
\begin{center}
\includegraphics[width=1.0\columnwidth]{figs/AllPermissions}
\caption{Permissions, sorted by how many apps request them in the entire GPStore dataset}
\label{fig:allpermissions}
\end{center}
\end{figure}

\begin{figure}[t]
\begin{center}
\includegraphics[width=1.0\columnwidth]{figs/AllPermissions_Tail}
\caption{Less commonly used permissions in the GPStore}
\label{fig:tailpermissions}
\end{center}
\end{figure}

Figure \ref{fig:allpermissions} shows all major permissions in the GPStore, sorted by frequency of use. This graph highlights several things: First off, \textit{INTERNET} is a dominant permission, with well over half of the GPStore apps requesting it. The next 5 permissions, \textit{Write to Storage}, \textit{Reading Phone Info}, \textit{Location Info} and \textit{Wake Lock}, all have over 50,000 apps that request them. A steep drop is seen for \textit{Read Contacts}, \textit{Call Phones}, and \textit{Camera}, with around 25,000 apps each. Of these top 9, 3 provide access to semi-personal information: \textit{Reading Phone Info} and the \textit{Location Info} permissions. It's only when we get to the lower 3 do we get access to personal information. 

Figure \ref{fig:tailpermissions} examines the tail permissions in the GPStore. This section contains the bulk of permissions related to PII, with \textit{Record Audio}, \textit{Read Calendar}, \textit{Read SMS}, and so on. It additionally contains many sensitive operations, like \textit{Send SMS} and \textit{System Alert} - which allows apps to draw windows over other apps. These permissions, while being a relatively small portion of the GPStore, still occupy a substantial portion. Indeed, with every dangerous permission, many legitimate use cases can be established, see Figure \ref{tab:permissionsanduses} for examples.

\begin{table*}[t]
\begin{small}
\begin{tabular}{p{3cm}|p{12.5cm}}
Permission & Use Case \\
\hline

\textit{Location (Fine)} & This one has a wide variety of uses, from location-specific news apps and games, to social networks. It's also used by ad networks included in many apps.  \\
\textit{Location (Coarse)} & This one follows the same trends as \textit{Location (Fine)}, but is skewed towards ad networks.  \\
\textit{Read Contacts} & Any access to the address book at all require this, so communication apps, social networks, or many apps that involve sharing with friends will use this.  \\
\textit{Call Phones} & This one is oddly popular. Many customization apps, especially those that seek to replace stock Android apps, will use this, especially if they are replacing address book or home screen functionalities. Additionally, many communication apps will use this, for obvious reasons.  \\
\textit{Camera} & Any photography app or video camera app will make heavy use of this permission. This functionality is often present in other apps as well, e.g. taking a photo of the user for use as a profile photo  \\
\textit{Write Contacts} & This permission is often used with \textit{Read Contacts}. Many social networking sites and services wish to provide ``contact syncing'' abilities with Android device, which requires having write-access to the Contacts database.  \\
\textit{Record Audio} & Like \textit{Camera}, audio apps and communication apps make heavy use of this.  \\
\textit{Send SMS} & Many apps seek to replace the default SMS app, which would therefore require all of the SMS related permissions.  \\
\textit{System Alert} & This permission protects drawing on the screen, on top of other apps. Many apps are designed to be on the screen at all times, either replacing Android components, or complementing them.  \\

\end{tabular}
\end{small}
%\vspace{-0.2in}
\caption{Use cases for common Android Permissions}
\label{tab:permissionsanduses}
%\vspace{-0.1in}
\end{table*}

Overall, we find many PII related permissions are requested a substantial number of times. A number of use cases exist, but it remains to be seen if the apps follow those use cases. To further examine this, we separate the apps in \temp{AndroidMarketDB} into the categories present on the GPStore, and rerun the analysis, seen in Figure \ref{fig:permissionspercategory}.








\section{Malware Dataset}





Ideally, all malicious behavior would show up in the permission fingerprint of an app, so therefore it's unique set of capabilties would stand out in the Google Play Store. Demonstrating that expected behavior and Permission Fingerprint are correlated will be an important part of that. Ultimately, however, we expect this method to have it's shortcomings.
